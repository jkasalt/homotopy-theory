\documentclass{scrartcl}
\usepackage[T1]{fontenc}
\usepackage{babel}
\usepackage{amsmath}
\usepackage{tikz-cd}

\title{Homotopy Theory -- Sheet 4 Exercise 5 Solution}
\author{Kaltenrieder, Garry \\
        \and
        Bracone, Luca}

\begin{document}
\maketitle
All maps are continuous and all spaces are based at $b \in B \subseteq A
	\subseteq X$ implicitly. The inclusions are denoted \[i_A: A \to X, i_B: B \to
	X, o_A: \{b\} \to A, o_B: \{b\} \to B, \text{ and } j: B \to A.\] From the
course, we have three long exact sequences, shown in the rows of the following
diagram:
\[
	\begin{tikzcd}
		&\pi_{n-1}(B) &\pi_{n-1}(A) &\vdots &\vdots & \\
		\dots \rar{} &\pi_n(A) \uar[blue]{ {\delta_B}_{|A} \circ (id, o_B)_*} \rar{{i_A}_*} &\pi_n(X) \uar[blue]{\delta_A \circ (id, o_A)_*} \rar{(id,{o_A})_*} &\pi_n(X,A) \uar[purple, swap]{\partial} \rar{\delta_A} &\pi_{n-1}(A) \uar[blue]{} \rar{} &\dots \\
		\dots \rar{} &\pi_n(B) \uar[blue]{} \rar{{i_B}_*} &\pi_n(X) \uar[blue]{} \rar{(id,{o_B})_*} &\pi_n(X,B) \uar[purple, swap]{(id, j)_*} \rar{\delta_B} &\pi_{n-1}(B) \uar[blue]{} \rar{} &\dots \\
		\dots \rar{} &\pi_n(B) \uar[blue]{} \rar{j_*}     &\pi_n(A) \uar[blue]{} \rar{(id,{o_B})_*} &\pi_n(A,B) \uar[purple, swap]{(i_A, id)_*} \rar{{\delta_B}_{|A}} &\pi_{n-1}(B) \uar[blue]{} \rar{} &\dots \\
		&\vdots \uar[blue] & \vdots \uar[blue] &\pi_{n+1}(X,A) \uar[purple, swap]{\partial} &\vdots \uar[blue] & \\
		& & &\vdots \uar[purple]{} & & \\
	\end{tikzcd}
\]
And we see that the sequence that we are searching for appears in the fourth
column of the diagram.  By choosing
the right maps (inclusions) for the blue arrows, it is clear enough that this
diagram is commutative. %TODO ???? actually I'm not sure
Let us prove exactness at\dots

\begin{itemize}
	\item[$\pi_n(X,B)$:]
		\begin{itemize}
			\item Let $\gamma \in \ker((id, j_*))$ then $\gamma$ is homotopic to the
			      constant path at $b$ relative to $A$ %TODO finish this
			\item Let $\gamma \in \Im((i_A, id)_*)$ then we have $\beta \in
				      \pi_n(A,B)$ such that $(i_A)_* \beta = \gamma$, but then we have that
			      $\Im(\gamma) \subset A$ and so $\gamma$ is homotopic to the constant
			      map relative to $A$
		\end{itemize}
	\item[$\pi_n(A,B)$:]
		\begin{itemize}
			\item %TODO finish this
			\item We know that $\partial = o_B \circ \delta_A$. Then, $i_A \circ o_B
				      \circ \delta_A = 0$ because the first row is exact.
		\end{itemize}
	\item[$\pi_n(X,A)$:]
		\begin{itemize}
			\item %TODO
			\item %TODO
		\end{itemize}
\end{itemize}
\end{document}
