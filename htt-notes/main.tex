\documentclass[main.tex]{subfiles}
\begin{document}

\chapter{Compact-Open Topology}

The aim of this chapter is to upgrade the set of (continuous) maps \(f: X \to Y\) to a space of \emph{maps}, \(\map(X,Y)\), by defining a topology on this set.
This makes the category of spaces \emph{enriched} over \(\Top\), as well as on \(\cat{Top_*}\)

\begin{rem}
	Here, \(X, Y, Z\) are all spaces and all maps are continuous.
\end{rem}

\begin{defn}
	The set of all maps \(f: X \to Y\) is given a topological structure with the \emph{compact-open} topology.
	It is given by the sub-basis
	
\[
		B(K, U) = \mset{f: X \to Y | f(K) \subset U}
	
\]
	for \(K\) compact in \(X\), and \(U\) open in Y.
\end{defn}

\begin{rem}
	Therefore, a basis for this topology is given by finite intersections of \(B(K_i, U_i), 1 \leq i \leq n\).
	Which are of the form
	
\[
		B(K_\bullet, U_\bullet) = \mset{f: X \to Y | f(K_i) \subset U_i, \forall 1 \leq i \leq n}.
	
\]
\end{rem}

\begin{rem}
	We will mostly work with finite CW-complexes.
	If \(Y\) is a CW-complex, then \(\map(X,Y)\) has the homotopy type of a CW-complex but this is not true if \(X\) is an infinite CW-complex.

\end{rem}

\begin{example}
	
\begin{enumerate}
		\item If \(X = \ast\), then \(\map(\ast, Y) \approx Y\).
		\item If \(X\) is discrete, say \(|X| = n\), then \(\map(X,Y) \approx Y^n \).
		\item if \(X = I\) an interval, then \(\map(X,Y)\) is the space of paths on \(Y\) with arbitrary starting and ending points.
		\item If \(X = S^1 \), then \(\map(S^1, Y)\) is called \(\Lambda Y\) the space of free loops.
	
\end{enumerate}
\end{example}

\begin{reminder}
	(\emph{Exponential law}) We denote by \(Y^X \) the set of set-theoretical maps \(f: X \to Y\).
	There is a natural bijection \(Z^{X \times Y} \simeq {\left(Z^{Y}\right)}^X \).
	It is given by
	
\begin{align*}
		f: (X \times Y) \to Z \mapsto a(f) = \varphi: X & \to Z^Y \\
		x & \mapsto f(x, -)
	
\end{align*}
	In fact, \(X \times -\) is left-adjoint to \(Z^- \).
	Inside \(Y^X \supset \map(X,Y)\), we ask if this adjunction is compatible with the compact-open topology.
	When do we have \(\map(X \times Y, Z) \simeq \map(X, \map(Y,Z))\)?
	This is not always the case, and we will have to impose restrictions.
\end{reminder}

\begin{prop}
	
\begin{enumerate}
		\item Let \(X\) be locally compact.
		 Then, the map
		
\begin{align*}
			 ev: \map(X,Y) \times X & \to Y \\
			 (f, x) & \mapsto f(x)
		
\end{align*}
		 is continuous.

		\item If \(Y\) is locally compact, a map \(f: X \times Y \to Z\) is continuous as a morphism of sets iff \(a(f): X \to \map(Y,Z)\) is continuous.
	
\end{enumerate}
\end{prop}

\begin{proof}
	
\begin{enumerate}
		\item Let \(U \ni f(x)\) be open in \(Y\). Since \(f\) is continuous, \(f\inv(U)\) is an open neighborhood of \(x\).
		 By assumption, \(X\) is locally compact, so there is \(x \in K \subset f\inv(U)\) with \(K\) compact and \(f(K) \subset U\).
		 We choose \(B(K,U) \in \map(X,Y)\) and \(K\) as neighborhood of \(x \in X\).
		 Then, \(ev(B(K,U) \times K) \subset U\), because \(ev\inv(U)\) is a neighborhood of \((f,x)\).

		\item Notice first that if \(f: X \times Y \to Z\) is a map, then \(f_{|x \times Y}\) is also continuous for all \(x \in X\).
		 Therefore \(a(f): X \to Z^Y \) actually lands in \(\map(Y,Z)\).
		 Next, we showthat \(a(f)\) is continuous by looking at the preimage of \(B(L,W)\) for \(L\) compact in \(Y\), and \(W\) open in \(Z\).
		 We see that
\[
a(f)\inv (B(K,W)) = \mset{x \in X | f(x \times L) \subset W}.
\]
		 Since W is open, \(f\inv(W)\) is open in \(X \times Y\) and contains \(x \times L\).
		 Let us choose an open subset, \(U \times V\) for the product topology in \(X \times Y\).
		 We claim we can assume that it contains \(x \times L\) because \(L\) is compact.

		 We have shown that every \(x \in a(f)\inv (B(K,W))\) has an open neighborhood in the preimage.
		 Finally, assume \(\varphi: X \to \map(X,Y)\) is continuous.
		 We prove that \(b(\varphi)\) is continuous:
		
\begin{align*}
			 b(\varphi), (X \times Y) \stackrel{\varphi \times 1}{\longrightarrow} & \map(Y,Z) \times Y \stackrel{ev}{\to} Z \\
			 (x,y) \mapsto & (\varphi(x), y) \mapsto \varphi(x)(y)
		
\end{align*}
		 We see that the map \(ev\) is continuous thanks to point one of this proof, as \(Y\) is locally compact.
		 Then, \(b(\varphi)\) is continuous since \(\varphi \times 1\) and \(ev\) are.
	
\end{enumerate}
\end{proof}

\begin{lemma}
	If X,Y are Hausdorff, then a sub-basis for \(\map(X \times Y, Z)\) is given by \(B(K \times L, W)\).
	Where \(K\) is compact in \(X\), \(L\) is compact in \(Y\), and \(W\) is open in \(Z\).
\end{lemma}

\begin{proof}
	Let \(A \subset X \times Y\) be compact and \(A_X = \pi_X(A)\) \(A_Y = \pi_Y(A)\) the projections.
	They are compact.
	For \(f \in B(A,W)\), in general, \(f \notin B(A_X \times A_Y, W)\).
	But, for any \((x,y) \in A\), \((x,y) \in A_X \times A_Y \cap f\inv(W)\).
	Choose \((x,y) \in U_x \times V_y\), where \(x \in U_x \subset A_X\) and \(y \in V_y \subset A_Y\), both open.
	Compact and Hausdorff implies normal, therefore regular, so we can separate \(x\) from \(A_X \setminus U_x\) by \(x \in U \subset \bar{U} \subset U_x\), and \(y\) from \(A_Y \setminus V_y\) in the same way.
	The compact subspace \(\bar U \times \bar V\) covers \(A\). So we have a finite cover, \(\bar{U_i} \times \bar{V_i}\).
	Then \(f \in B(\bar{U_i} \times \bar{V_i}, W)\) since \(\bar{U_i} \times \bar{V_i} \subset A\).
	Finally, we have \(f \in \bigcap_{i = 1}^n B(\bar{U_i} \times \bar{V_i}, W)\)
\end{proof}

\begin{rem}
	Notice that \(a(-)\) sends \(B(K \times L, W)\) to \(B(K, B(L,W))\)
\end{rem}

\begin{lemma}
	If a space \(Z\) is Hausdorff, \(\map(Y,Z)\) is as well.

\end{lemma}

\begin{proof}
	exercise

\end{proof}

\begin{lemma}
	Let \(Z\) be Hausdorff, and \(\cB\) be a sub-basis of its topology.
	Then, the sets of the form \(B(K, \cB)\) for \(K\) compact in \(Y\), form a sub-basis for the compact-open topology on \(\map(Y,Z)\)
\end{lemma}

\end{document}
