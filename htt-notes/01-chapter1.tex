\documentclass[main.tex]{subfiles}
\begin{document}

\chapter{Compact-Open Topology}

The aim of this chapter is to upgrade the set of (continuous) maps \(f: X \to Y\) to a space of \emph{maps}, \(\map(X,Y)\), by defining a topology on this set.
This makes the category of spaces \emph{enriched} over \(\Top\), as well as on \(\cat{Top_*}\)

\begin{rem}
	Here, \(X, Y, Z\) are all spaces and all maps are continuous.
\end{rem}

\begin{defn}
	The set of all maps \(f: X \to Y\) is given a topological structure with the \emph{compact-open} topology.
	It is given by the sub-basis
	\[
		B(K, U) = \mset{f: X \to Y | f(K) \subset U}
	\]
	for \(K\) compact in \(X\), and \(U\) open in Y.
\end{defn}

\begin{rem}
	Therefore, a basis for this topology is given by finite intersections of \(B(K_i, U_i), 1 \leq i \leq n\).
	Which are of the form
	\[
		B(K_\bullet, U_\bullet) = \mset{f: X \to Y | f(K_i) \subset U_i, \forall 1 \leq i \leq n}.
	\]
\end{rem}

\begin{rem}
	We will mostly work with finite CW-complexes.
	If \(Y\) is a CW-complex, then \(\map(X,Y)\) has the homotopy type of a CW-complex but this is not true if \(X\) is an infinite CW-complex.

\end{rem}


\begin{example}
	\begin{enumerate}
		\item If \(X = \ast\), then \(\map(\ast, Y) \approx Y\).
		\item If \(X\) is discrete, say \(|X| = n\), then \(\map(X,Y) \approx Y^n \).
		\item if \(X = I\) an interval, then \(\map(X,Y)\) is the space of paths on \(Y\) with arbitrary starting and ending points.
		\item If \(X = S^1 \), then \(\map(S^1, Y)\) is called \(\Lambda Y\) the space of free loops.
	\end{enumerate}
\end{example}

\begin{reminder}
	(\emph{Exponential law}) We denote by \(Y^X \) the set of set-theoretical maps \(f: X \to Y\).
	There is a natural bijection \(Z^{X \times Y} \simeq {\left(Z^{Y}\right)}^X \).
	It is given by
	\begin{align*}
		f: (X \times Y) \to Z \mapsto \alpha(f) = \varphi: X & \to Z^Y         \\
		x                                                    & \mapsto f(x, -)
	\end{align*}
	In fact, \(X \times -\) is left-adjoint to \(Z^- \).
	Inside \(Y^X \supset \map(X,Y)\), we ask if this adjunction is compatible with the compact-open topology.
	When do we have \(\map(X \times Y, Z) \simeq \map(X, \map(Y,Z))\)?
	This is not always the case, and we will have to impose restrictions.
\end{reminder}

\begin{prop}
	\begin{enumerate}
		\item Let \(X\) be locally compact.
		      Then, the map
		      \begin{align*}
			      ev: \map(X,Y) \times X & \to Y        \\
			      (f, x)                 & \mapsto f(x)
		      \end{align*}
		      is continuous.

		\item If \(Y\) is locally compact, a map \(f: X \times Y \to Z\) is continuous as a morphism of sets iff \(\alpha(f): X \to \map(Y,Z)\) is continuous.
	\end{enumerate}
\end{prop}

\begin{proof}
	\begin{enumerate}
		\item Let \(U \ni f(x)\) be open in \(Y\). Since \(f\) is continuous, \(f\inv(U)\) is an open neighborhood of \(x\).
		      By assumption, \(X\) is locally compact, so there is \(x \in K \subset f\inv(U)\) with \(K\) compact and \(f(K) \subset U\).
		      We choose \(B(K,U) \in \map(X,Y)\) and \(K\) as neighborhood of \(x \in X\).
		      Then, \(ev(B(K,U) \times K) \subset U\), because \(ev\inv(U)\) is a neighborhood of \((f,x)\).

		\item Notice first that if \(f: X \to Y\) is a map, then \(f_{}\)
                        asdjasd
    adsad

	\end{enumerate}
\end{proof}


\end{document}
